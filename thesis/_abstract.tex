\chapter*{Abstract}

One of the fundamental questions of machine learning in particle
physics is whether we can improve the resolution of our measuring
instruments. Is it possible to gain deeper insights from the
structure? This question is examined in this thesis using the example
of calorimeter data and jet analysis. Different types of networks will
be investigated. The technique of the loss function design was
examined, which allows to reduce systematics in the data
structure. Overall, it is shown that deep learning can achieve better
performance in the application cases.

\chapter*{Zusammenfassung}
Eine der grundlegenden Fragen des Machinellen Lernens in der
Teilchenphysik ist ob wir die Auflösung unser Messinstrumente
verbessern können. Ist es möglich aus der Struktur tiefere
Erkenntnisse zu gewinnen. Diese Fragestellungen werden in dieser
Arbeit am Beispiel von Calorimeter Daten und der Jet Analyse
betrachtet. Es werden unterschiedliche Netzwerktypen
ausgetestet. Hierbei wurden Technik des Loss Funktionsdesigns erprobt,
die es erlauben Systematiken in der Datenstruktur zu
vermindern. Insgesamt wird dargestellt, das mittels Deep Learning sich
eine bessere Performance in den Anwendungsfällen erzielen lässt.
