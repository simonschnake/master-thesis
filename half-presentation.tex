\documentclass[10pt]{beamer}
\usetheme{m}                     % Use metropolis theme
\usepackage{booktabs}
\usepackage[scale=2]{ccicons}
\usepackage{pgfplots}
\usepackage[ngerman]{babel}
\usepackage[utf8]{inputenc}
\usepackage{blindtext}
\usepackage{amsmath}
\usepackage[italicdiff]{physics}
\usepackage[italic]{hepnames}
\usepackage{graphicx}
\usepackage{float}
\usepackage{color}
\usepackage{physics}

\usepgfplotslibrary{dateplot}


%Frontpage
\title{supersymmetrie}
\date{\today}
\author{Simon Schnake}
\institute{Universität Hamburg}

%Document
\begin{document}
  \maketitle
  
  \begin{frame}{Inhalt}
\begin{itemize}[<+- | alert@+>]
	\item Standardmodell
	\item Supersymmetrie (SUSY)
	\item Minimal Supersymmetric Standard Model
	\item Suche nach Supersymmetrie am Large Hadron Collider (LHC)
\end{itemize}
\end{frame}
  
  
\begin{frame}{das Standardmodell}
	\begin{figure}[htp]
    	\includegraphics[scale=0.25]{Standardmodell.pdf}
    \end{figure}
\end{frame}  

\begin{frame}{Anforderungen an das Standardmodell}
  \begin{columns}[c,onlytextwidth]
    \column{0.5\textwidth}
    	\metroset{block=fill}
		\begin{block}{Materie}<1->
		Der fundamentale Aufbau der Materie im Universum soll beschrieben werden.
		\end{block}
		\begin{block}{Kräfte}<2->
		Alle Kräfte der Natur sollen beschrieben werden.
		\end{block}
		\begin{block}{Extrapolierbarkeit}<3->
		Die Theorie soll auch bei höheren Energien oder kleineren Abständen gelten.
		\end{block} 
    \column{0.5\textwidth}
		\begin{figure}[htp]
    	\includegraphics[scale=0.25]{Standardmodell.pdf}
    	\end{figure}
  \end{columns}
\end{frame}


\begin{frame}{Das Standardmodell im Test}
  \begin{columns}[c,onlytextwidth]
    \column{0.5\textwidth}
    	\metroset{block=fill}
		\begin{block}{Materie}<1->
		\alert<1>{Im Labor messbaren Teilchen - Ja! \\ Dunkle Materie - Nein!}
		\end{block}
		\begin{block}{Kräfte}
		\only<2->{\alert<2>{Elektroschwache und Starke Wechselwirkungen werden beschrieben. Gravitation - Nein!\\ Auch keine Vereinigung aller Kräfte}}
		 \only<1> {Alle Kräfte der Natur sollen beschrieben werden.}
		\end{block}
		\begin{block}{Extrapolierbarkeit}
		\only<1-2>{Die Theorie soll auch bei höheren Energien oder kleineren Abständen gelten.}
		\only<3->{\alert<3>{Hierachie-Problem des Standardmodells ist nicht gelöst.}}
		\end{block} 
    \column{0.5\textwidth}
		\begin{figure}[htp]
    	\includegraphics[scale=0.25]{Standardmodell.pdf}
    	\end{figure}
 \end{columns}
\only<4>{\textbf{Das Standardmodell reicht nicht zur Beschreibung des Universums!}}
\end{frame}
	

%  \begin{frame}{Hierachie-Problem des Standartmodells}
%		Hier noch das Hierarchie-Problem erläutern
%  \end{frame}  
	
\begin{frame}{Was ist Supersymmetrie?}
	\begin{columns}[c,onlytextwidth]
    	\column{0.6\textwidth}
    		\metroset{block=fill}
    		\begin{block}{Prinzip}<1->
    			SUSY ist ein Prinzip, aus dem Theorien abgeleitet werden können.
    		\end{block}
    		\begin{block}{Materie und Kraft}<2->
    			\centering
    			$\boxed{\begin{array}[c]{c} \text{GLEICHUNG} \\ \textcolor{magenta}{\only<3-7>{\text{MATERIE}}\only<2,8->{\text{FERMION}}} \end{array}} = \boxed{\begin{array}[c]{c} \text{GLEICHUNG} \\ \textcolor{red}{\only<3-7>{\text{KRAFT}}\only<2,8->{\text{BOSON}}} \end{array}}$ \\
    		\end{block}
    		
    		\only<4-7>{
    			${\mathcal{L}}= \textcolor{magenta}{\text{MATERIE}} + \textcolor{red}{\text{KRAFT}}$
    		}
    		\only<5-7>{
    			$\mathbb{SUSY}({\mathcal{L}})= \textcolor{magenta}{\text{KRAFT}} + \textcolor{red}{\text{MATERIE}}$
    		}
    		\only<6-7>{
    			$\mathcal{L}= \textcolor{magenta}{\text{KRAFT}} + \textcolor{red}{\text{MATERIE}} + \textcolor{magenta}{\text{MATERIE}} + \textcolor{red}{\text{KRAFT}}$
    		}
    		\only<7>{
    			$\mathbb{SUSY}(\mathcal{L})= \textcolor{magenta}{\text{MATERIE}} + \textcolor{red}{\text{KRAFT}} + \textcolor{magenta}{\text{KRAFT}} + \textcolor{red}{\text{MATERIE}}$
    		}
    		\begin{block}{SUSY Algebra}<8->
    			$Q_a \ket{F} = \ket{B} \ , \quad Q_a \ket{B}=\ket{F} $
    		\end{block}
    		\begin{block}{Partnerteilchen}<9->
    			Zu jedem Teilchen des Standardmodells, gibt es supersymmetrische Partnerteilchen.
    		\end{block}
    	\column{0.4\textwidth}
    		\only<1-8>{\begin{figure}[htp]
    	\includegraphics[scale=0.5]{susyparticles2.pdf}
    	\end{figure}}
    		\only<9->{\begin{figure}[htp]
    	\includegraphics[scale=0.5]{susyparticles1.pdf}
    	\end{figure}}
    \end{columns}
\end{frame}

\begin{frame}{Wo zu Supersymmetrie?}
	\metroset{block=fill}
	\begin{block}{Vereinigung der Kräfte}<1->
	In der Supersymmetrie vereinigen sich die elektromagnetische,\\ schwache und starke Wechselwirkung bei hohen Energien.
	\end{block}
	\begin{block}{Existenz des Higgs-Boson}<2->
	Das Hierachie-Problem wird gelöst.
	\end{block} 
	\begin{block}{Dunkle Materie}<3->
	Das leichteste (stabile) supersymmetrische Teilchen (LSP) ist ein guter Kanidat für dunkle Materie.
	\end{block}
\end{frame}

\begin{frame}{MSSM - Minimal Supersymmetric Standard Model}
Das Minimal Supersymmetric Standard Model ist die supersymmetrische Variante des Standardmodells, also besitzt es auch dieselbe Eichwechselwirkung.
\begin{equation*}
G = \mathrm{SU}(3)_C\otimes \mathrm{SU}(2)_L \otimes \mathrm{U}(1)_Y
\end{equation*}
Der Feldinhalt wird um ein weiteres Higgs-Dublett erweitert und anschließend jedem Feld/Teilchen genau ein Superpartner zugeordnet.

Im MSSM ist das LSP (leichteste supersymmetrisches Teilchen) das Neutralino $\PSneutralinoOne$, welches deshalb der dunkle Materie Kanidat ist.

\end{frame}

\begin{frame}{GUT - grand unified theory}
%\footnotesize{W. de Boer - Grand Unified Theories and Supersymmetry in -  Particle Physics and Cosmology hep-ph/9402266 March, 1994}
 \begin{figure}[htp]
    	\includegraphics[scale=0.4]{GUT.pdf}
 \end{figure}
\end{frame}

\begin{frame}{SUSY Suche am LHC}

\begin{figure}[htp]
	\includegraphics[width=0.75\textwidth]{decay.pdf}
\end{figure}

\metroset{block=fill}
\begin{block}{Experimentelle Signatur}
\begin{itemize}[<+- | alert@+>]
	\item Mehrere Jets
	\item Fehlender transversaler Impuls
	\item Keine isolierten Leptonen
\end{itemize}
\end{block}
\end{frame}

\begin{frame}{SUSY Suche am LHC}

\begin{columns}[c,onlytextwidth]
    \column{0.5\textwidth}
    \begin{figure}[htp]
		\includegraphics[width=\textwidth]{barrel1.pdf}
	\end{figure}
	\column{0.5\textwidth}
    \begin{figure}[htp]
		\includegraphics[width=\textwidth]{barrel2.pdf}
	\end{figure}
\end{columns}
\end{frame}

\begin{frame}{SUSY Suche am LHC}

\begin{figure}[htp]
	\includegraphics[width=0.6\textwidth]{plot.pdf}
\end{figure}
\end{frame}

\begin{frame}{SUSY Suche am LHC}

\begin{figure}[htp]
	\includegraphics[angle=-90,width=0.7\textwidth]{results.pdf}
\end{figure}
\end{frame}

\begin{frame}{Zusammenfassung}
\begin{itemize}[<+- | alert@+>]
	\item Das Standardmodell ist unvollständig
	\item Supersymmetrie ist eine Theorie die jedem Teilchen ein Partnerteilchen mit unterschiedlichem Spin gibt
	\item Supersymmetrische Theorien können das Standardmodell erweitern
	\item Supersymmetrie ist schwer zu messen
	\item Bis jetzt wurden Bereiche für Supersymmetrie ausgeschlossen, aber keine Supersymmetrie gemessen
\end{itemize}
\end{frame}

\end{document}
